\tolerance=10000

\documentclass[conference]{./IEEEtran/IEEEtran}
\usepackage{hyperref}       % hyperlinks
\usepackage{url}            % simple URL typesetting
\usepackage{breakurl}
\usepackage{booktabs}       % professional-quality tables
\usepackage{amsfonts}       % blackboard math symbols
\usepackage{nicefrac}       % compact symbols for 1/2, etc.
\usepackage{listings}
\usepackage{amssymb}
\usepackage{amsmath}
\usepackage{eqnarray}
\usepackage{mathtools}
\usepackage{lipsum}
\usepackage{adjustbox}
\usepackage{booktabs}
\usepackage{multirow}
\def\UrlBreaks{\do\/\do-}

\newcommand*\rot[1]{\rotatebox[origin=c]{90}{#1}}

\usepackage{ifthen}
\usepackage{color}
\usepackage{xcolor}
\usepackage[]{algorithm}
\usepackage{clrscode4e}
\usepackage[small]{caption}
\usepackage{subcaption}
\usepackage{url}

\captionsetup[algorithm]{font=footnotesize}

\newboolean{showcomments}
\setboolean{showcomments}{true}

\ifthenelse{\boolean{showcomments}}
           { \newcommand{\mynote}[3] {
               \fbox{\bfseries\sffamily\scriptsize#1}
                    {\small$\blacktriangleright$\textsf{\emph{\color{#3}{#2}}}$\blacktriangleleft$}}}
           { \newcommand{\mynote}[3]{}}

\newcommand{\aleks}[1]{\mynote{Aleks}{#1}{violet}}
\newcommand{\seung}[1]{\mynote{Seung}{#1}{blue}}
\newcommand{\kisuk}[1]{\mynote{Kisuk}{#1}{green}}

\newcommand{\remove}[1]{}


\DeclarePairedDelimiter{\ceil}{\lceil}{\rceil}
\DeclarePairedDelimiter{\floor}{\lfloor}{\rfloor}
\DeclarePairedDelimiter{\angled}{\langle}{\rangle}
\DeclarePairedDelimiter{\sqb}{\big[}{\big]}


\begin{document}

\special{papersize=8.5in,11in}
\setlength{\pdfpageheight}{\paperheight}
\setlength{\pdfpagewidth}{\paperwidth}

\title{Compile--Time Optimized and Statically Scheduled N--D ConvNet
  Primitives for Multi--Core and Many--Core (Xeon Phi) CPUs }


\author{\IEEEauthorblockN{Aleksandar Zlateski\IEEEauthorrefmark{1}, H
    Sebastian Seung\IEEEauthorrefmark{2}} \IEEEauthorblockA{Princeton
    Neuroscience Institute\\ and Computer Science
    Department\\ Princeton University\\ Princeton, NJ 08540
    USA\\ \IEEEauthorrefmark{1}{\tt zlateski@princeton.edu},
    \IEEEauthorrefmark{2}{\tt sseung@princeton.edu}}} \maketitle



\begin{abstract}

  Convolutional networks (ConvNets), largely running on GPUs, have
  become the most popular approach to computer vision.  Now that CPUs
  are closing the FLOPS gap with GPUs, efficient CPU algorithms are
  becoming more important.  We propose a novel parallel and vectorized
  algorithm for N--D convolutional layers.  Our goal is to achieve
  high utilization of available FLOPS, independent of ConvNet
  architecture and CPU properties (e.g. vector units, number of cores,
  cache sizes). Our approach is to rely on the compiler to optimize
  code, thereby removing the need for hand-tuning.  We assume that the
  network architecture is known at compile--time. Our serial algorithm
  divides the computation into small sub--tasks designed to be easily
  optimized by the compiler for a specific CPU.  Sub--tasks are
  executed in an order that maximizes cache reuse.  We parallelize the
  algorithm by statically scheduling tasks to be executed by each
  core.  Our novel compile--time recursive scheduling algorithm is
  capable of dividing the computation evenly between an arbitrary number
  of cores, regardless of ConvNet architecture.  It introduces
  zero runtime overhead and minimal synchronization overhead.  We
  demonstrate that our serial primitives efficiently utilize available
  FLOPS (75-95\%), while our parallel algorithm attains 50-90\%
  utilization on 64+ core machines. Our algorithm is competitive with
  the fastest CPU implementation to date (MKL2017) for 2D object
  recognition, and performs much better for image segmentation.
  For 3D ConvNets we demonstrate comparable performance to the
  latest GPU hardware and software even though the CPU is only capable
  of half the FLOPS of the GPU.

\end{abstract}
\setlength{\belowcaptionskip}{-10pt}

\section{Introduction}

  ConvNets had become... popular \mynote{seung}{intro about convnets}.
  The argument of using CPUs or GPUs for deep learning had reached
  almost religious proportions.  The GPUs, which are capable of more
  FLOPs/s seem to win the battle by a large margin -- disproportional
  to the FLOPs/s available to the two architectures.  We find this
  absurd, as the CPUs have been around for much longer, and the CPU
  programming model changes very little from a generation to the next.
  On the other side, GPUs seem to change a lot.

  As the CPUs are closing the gap in terms of FLOPs/s performance.
  Xeon Phi Knights Landing chips are capable of up to
  6TFLOPs/s~\cite{}.  Upcoming Skylake server CPU's will be packing
  $3.2$ TFLOPs/s per single chip, delivering up to $25.8$ TFLOPs/s on
  an 8-way configuration~\cite{}.  We believe that the main battle
  should be in harvesting the computational power of the hardware.  We
  are surprised to see that the GPU implementations have higher
  utilization of the \% FLOPs/s available, compared to the CPU.

  The amount of computation required by the convolutional layers of a
  ConvNet grows as a square of the number of channels (images).  In
  contrast, other types of layers require linear amount of operations
  per output image.  State of the art networks have convolutional
  layers with anywhere from $64$ to $1024$ images.  This means that
  $98.5\%$ up to $99.9\%$ of the computation is in the convolutional
  layers.  It is therefore most important to speed up these layers.

  Reducing convolution to matrix multiplication has been a popular
  approach due to the availability of highly efficient matrix
  multiplication routines.  This approach has, however, both a
  computational and memory overhead due to the requirement of creating
  in--memory matrices.  Caffe con Troll (CCT)~\cite{hadjis2015shallow}
  uses blas to perform convolutions -- we will compare us to them.

  Initially, cuDNN~\cite{chetlur2014cudnn} took that approach,
  however, it has since implemented more efficient direct convolution
  algorithms.

  Convolutional layers of a ConvNet are computationally dominating
  both training and inference.  The forward and backward pass involve
  a convolution of an image with a small kernel.  Typically ``valid''
  dense convolution, possibly with strides is performed.  More
  advanced types of convolutions can decomposed to a series of dense
  convolutions~\cite{szegedy2015going}.

  We propose a new approach.  We harvest the power of C++ templates to
  create compile time primitives by harvesting the compiler.  Given
  the specifications of the CPU, our meta--algorithms optimize and
  vectorize single threaded code, as well as generate static
  scheduling for parallelizing the computation over multiple cores.

  We show that our approach can utilize extremely high percentage of
  the available FLOPS regardless of the CPU generation, as well as
  have near linear scalability over the number of cores available.
  The performances of our algorithms are competitive with the state of
  the art hand optimized code for specific architectures.  Our code
  yields better performances on the KNL compared to the state of the
  art GPU implementations for 3D convolutional networks, even though
  KNL has less FLOPS/s available.

  We further demonstrate the power of our approach for inference only
  computation.

  Even though we provide an efficient, publicly available, open source
  implementation for popular ConvNet primitives, the goal of this
  paper is to point to the right direction for solving the problem,
  rather than giving the final ``optimized'' implementation.

  Efficient algorithms for training and inference of ConvNets with
  larger kernels have been introduced in
  ~\cite{zlateski2016znn,zlateski2016znni}, however the evidence
  suggests that ConvNets with smaller kernels, and larger number of
  layers perform better on nearly all problems. (Cite).

  %% \subsection{GPUs vs CPUs}

  %% GPUs have streaming multiprocessors, while the CPUs have in--order,
  %% memory coherent cores.

  %% GPUs model is so called SIMT (single instruction multiple
  %% threads)... dwell on it..

  %% CPUs on the other side have a complex structure of many cores and
  %% virtual cores, many cache layers and each core capable of performing
  %% SIMD instructions on up to 16 floating point numbers.

  %% In this paper we are not trying to answer the GPU vs CPU question,
  %% but rather propose an algorithm and implementation for ConvNet
  %% primitives that can utilize very high percentage of the theoretical
  %% peak performance of both many--core (Xeon Phi) and multi--core
  %% (Xeon) CPUs.  As shown later, our algorithm performs well on
  %% multi--chip systems as well.

  %% \subsection{Mulit--core vs Many--core}

  %% Explain the differences NUMA, caches, etc..


  \aleks{Summarize and cite the text below}
  A popular convolution implementation is to unroll the data until the
  computation is in the form of a large matrix multiplication
  (Chellapilla et al. (2006)). This is the strategy followed by many
  implementors, since matrix multiplication is a well-tuned linear
  algebra primitive available on virtually any platform. While it is
  possible to provide instances of direct calculation that are faster
  than matrix unrolling (e.g., for large S, Krizhevsky (2014)), it is
  challenging to provide an implementation that is faster for more
  than just a small subset of possible convolution problems.

\section{Convolutional layers}

  We begin by describing the computation performed during the forward,
  backward and the update pass of a convolutional layer.

  In the forward pass of a convolutional layer a tuple of $f$ images
  are transformed into another tuple of $f'$ images.  We want to
  process a batch of $S$ inputs to yield a batch of $S$ outputs, via
  \[
  O_{s,j} = \sum_{i=1}^f w_{ji}\ast I_{s,i}
  \]

  for $1 \le s \le S$ and $1 \le j \le f'$.  Here $I_{s,i}$ is the
  $i^{th}$ image of the $s^{th}$ input in the batch, and $O_{s,j}$ is
  the $j^{th}$ image of the $s^{th}$ output in the batch, and $w_{ji}$
  is the kernel from the $i^{th}$ image in an input tuple to the
  $j^{th}$ image in an output tuple.


  And for the backward pass

  \[
  \frac{\partial L}{\partial I_{b,i}} = \sum_{j=1}^{f'}
  \frac{\partial L}{\partial O_{b,j}} \star w_{ji}
  \]

  Update

  \[
  \frac{\partial L}{\partial w_{j,i}} = \sum_{b=1}^B I_{b,i} \ast
    \frac{\partial L}{\partial O_{b,j}}
  \]

  We will assume 3D images and kernels.  If $I_{s,i}$ has size
  $\vec{n} = \angled{n_x, n_y, n_z}$ and $w_{ji}$ has size $\vec{k} =
  \angled{k_x,k_y,k_z}$, then we can regard $I$ as a 5D tensor of size
  $S \times f \times n_x \times n_y \times n_z$, $w$ as a 5D tensor of
  size $f' \times f \times k_x \times k_y \times k_z$, and $O$ as a 5D
  tensor of size $S \times f' \times n_x' \times n_y' \times n_z'$,
  where $\vec{n}' = \vec{n} - \vec{k} + \vec{1}$.

  We will refer to the sizes of the 5D tensors $I$ and $O$ as input
  and output shape, respectively.  The relationship between input
  shape and output shape depends on kernel size as in
  Table~\ref{table:layers_complexity}.


\subsection{Data layout}

  Both the input and output of both the forward pass and backward pass
  of an $N$--dimensional convolutional layer can be represented as
  $(N+2)$ dimensional tensor.  The pixel at $(x_1,x_2,\dots,x_N)$ of
  $c^{th}$ image in the batch $b$ is located at
  $(b,c,x_1,x_2,\dots,x_N)$

  Similarly, the kernels can also be represented as $(N+2)$
  dimensional tensor, where the $(x_1,x_2,\dots,x_N)$ element of the kernel

  To efficiently use FMA instructions we use data layout as proposed
  in~\cite{jeffers2016knl, cpu-myth}, except that we generalize it for
  N--dimensional case.

  Instead of using $(N+2)$ dimensional tensor for the input/output
  images we use a $(N+3)$ dimensional tensor such that the element
  $(b,c,x_1,x_2,\dots,x_N)$ of the original tensor is mapped to an
  element $(b,\floor{c/S},x_1,x_2,\dots,x_N,c \mod S + 1)$.  This new
  tensor size of $B \times \floor{C/S} \times X_1 \times X_2 \times
  \dots \times X_N \times S$.

  $S$ denotes the length of the CPU's SIMD register in terms of number
  of floats it can store.  This is one of very few parameters required
  to specify when compiling our code for different CPUs.  In the case
  of Haswell and Skylake $S$ equals to $8$ and the Knights Landing has
  $S=16$.  Note that when $C$ is not divisible by $S$, the new tensor
  will have more elements, and thus require more memory than the
  original tensor.  However, nearly all modern ConvNets~\aleks{Cite
    alex-net, googlelenet, oxfordnet} have $C$ as a multiple of $16$.

\section{Static scheduling}

  The common underlying theme of our forward/backward propagation
  algorithm as well as update algorithms is static scheduling of the
  work for each available core.  In order to fully utilize $N$ cores,
  we need to divide the work into exactly $N$ independent parts that
  perform roughly the same number of operations.  Ideally all the
  cores would execute the same code and have the same memory access
  patterns.  This approach relies on having exclusive access to all
  the cores.  This is reasonable as one can limit the number of cores
  used for the ConvNet computation, while leaving some cores to the
  system.

  There are multiple reasons why we choose this approach over the
  alternative, dynamic scheduling.  To efficiently utilize many cores
  with a dynamic scheduling approach we would have to split the
  problem into many fine granularity tasks.  As the number of threads
  increases, maintaining a synchronized global queue of available
  tasks becomes more expensive.  Having multiple local queues and a
  work--stealing dynamic
  scheduler~\cite{reinders2007intel,willhalm2008putting} can only
  partially lower the synchronization overhead.  Additionally, with a
  dynamic scheduling we don't have control over the memory access
  patterns of each of the cores, as any core can possibly execute any
  task.  This can yield a large overhead due to cache misses,
  especially on multi--chip machines, where using L3 cache is crucial
  due to NUMA~\footnote{Non--uniform memory access}.

  \subsection{Implementation details}

  We provide custom fork--join primitive.  The main thread of the
  program is pinned to the core $0$ as described
  in~\cite{jeffers2015high}. Additional $N-1$ threads are spawned and
  pinned to cores $1$ to $N-1$.  The main thread continues with the
  execution of the program, while the other threads block on a single
  barrier.  On a CPU with $C$ cores and $H$ hyper--threads, $h^{th}$
  hyper--thread of the $c^{th}$ core executes the thread number $c
  \times H + h$, Note that this differs from the standard unix thread
  assignment where the same core/hyper--thread is assigned to the
  system thread with the id of $h * H + c$.  We choose this thread
  assignment in order to have local threads execute on local, and
  possibly the same physical core.  For multi--chip system, local
  threads will be assigned to the same physical chip.

  On fork, threads $1$ to $N-1$ are unblocked to execute tasks $1$ to
  $N-1$, and the calling thread executes the task $0$.  After the
  execution of the task, each thread waits on a single barrier.  When
  all threads complete their task, the main thread continues with the
  program execution, while the remaining $N-1$ threads block.  Our
  implementation uses \texttt{pthreads}.

\section{Problem definition and motivation}

  Each element $I'(b,f',n_x',n_y',n_z')$ of the tensor computed via
  {\footnotesize
  \begin{equation}
  \sum_{f} \sum_{k_x} \sum_{k_y} \sum_{k_z}
  I(b,f,n_x+k_x,n_y+k_y,n_z+k_z) \cdot W(f',f,k_x,k_y,k_z)
  \label{eqn:forward}
  \end{equation}
  } Computing all the output images can be easily implemented as 9
  nested loops over $b,f',n_x',n_y',n_z',f,k_x,k_y,k_z$, with an
  arbitrary nesting order.  In the innermost loop, one multiplication
  and one addition is performed.  Thus, we need total of
  $2BFF'N_x'N_y'N_z'K_zK_yK_z$ floating point operations (FLOPs) to
  compute the result.  We implemented such approach in C++ and
  measured the speed of the implementation in GFLOPS (Giga FLOPS per
  second) by dividing the FLOPs required by the time taken to perform
  the computation.  While running on a CPU core capable of $80$
  GFLOPS, the average measured speed was $0.87$ GFLOPS -- less than
  $1.1\%$ utilization.  The code was compiled using all optimization
  switches including the ones enabling AVX2 and FMA that were
  available.

  This come as no surprise, as modern CPUs can achieve high
  performances only under very special circumstances.  Mainly, each
  core of modern CPUs has up to two FMA vector units each of which is
  capable of performing $2S$ floating point instructions per cycle via
  a fused multiply--add operation ($y = a\cdot x + b$).  Here, $S$ is
  the width of the vector register (how many floating point numbers
  fit inside the register).  The peak performance of a single core is
  then reached when, in each cycle, all available FMA units perform a
  fused multiply--add operation with the input stored in the register
  file.  Additionally, each CPU has a different latency of the FMA
  units.  This means that the result of the computation is not
  immediately available, but rather after $l$ cycles, where $l$ is the
  latency of the unit.  To fully utilize the $n$ FMA units, a program
  must continuously issue FMA instructions such that no $nc$
  consecutive instructions are dependent (a result of one instruction
  is an input for some other one).  Modern compilers are aware of
  these limitations, and will try to assemble the code in the most
  optimal way.

  Loading data from memory to a register introduces an additional
  overhead.  Thus, to achieve high performance, one must minimize such
  data transfers and re--use data in the register file as much as
  possible.  Furthermore, to speedup the loading of data from memory
  to the register requires the data to be properly aligned in memory.

  Our first motivation comes from the fact that there is a room of $90
  \times$ improvement of the serial algorithm over the naive approach.
  A bit about the solution for the serial algo.

  Efficiently utilizing CPUs with multiple cores introduce an
  additional constraint.  In order to fully utilize all available
  cores, they must perform independent computation and minimize
  synchronization.  A core shouldn't wait for a result computed by
  another core.

  Because of these constraints, the naive way of computing
  convolutions is very inefficient.  Designing an algorithm that
  follows all the constraints is challenging.  To design such
  algorithm, we are motivated by the following two principles.  First,
  we divide computation into small sub--tasks.  We provide a
  relatively simple implementation for each task so that the compiler
  can generate optimal machine code.  Secondly, the computation will
  be evenly distributed among the available cores, such that each core
  does an equal amount of work.  This minimal synchronization is
  necessary and each core can start and ends the computation at the
  same time.

  A bit about the motivation and solution of the parallel algo

\section{Data layout and notation}

  To efficiently use FMA instructions we generalize the data layout as
  proposed in~\cite{jeffers2016knl, cpu-myth} to the $N$--dimensional
  case.  {\bf We encourage the reader to get a good understanding of
    our data layout before proceeding to the next section.}

  All the tensors are stored as row--major multi--dimensional, packed
  an properly aligned arrays in memory.  To distinguish memory arrays
  from tensors we will use lower letters to refer to them, and C++
  like syntax to refer to a particular element.

  {\bf Image tensors} are stored as 6--dimensional
  arrays in memory.  An image tensor of size $B \times F \times X
  \times Y \times Z$ is stored as an array of size $B \times
  \ceil{F/S} \times X \times Y \times Z \times S$.  Here $S$ is the
  size of the CPU's vector register \aleks{details}.  A tensor element
  $I(b,f,x,y,z)$ corresponds to the array element
  $i\sqb{b}\sqb{\floor{f/S}}\sqb{x}\sqb{y}\sqb{z}\sqb{f\mod S}$

  We maintain two copies of a {\bf kernel tensor} in memory -- a
  regular and reflected form.  For a kernel tensor of size $F' \times
  F \times K_x \times K_y \times K_z$ both memory arrays have size
  $\ceil{F'/S} \times \ceil{F/S} \times K_X \times K_Y \times K_Z
  \times S \times S$.  A tensor element $W(f',f,k_x,k_y,k_z)$ is
  stored in $w\sqb{\floor{f'/S}}
  \sqb{\floor{f/S}}\sqb{k_x}\sqb{k_y}\sqb{k_z}\sqb{f\mod S}\sqb{f'
    \mod S}$ in one of the arrays, and in $w^T\sqb{\floor{f'/S}}
  \sqb{\floor{f/S}}\sqb{K_x-k_x}\sqb{K_y-k_y}\sqb{K_z-k_z}\sqb{f'\mod
    S}\sqb{f\mod S}$

  Whenever a kernel tensor is updated, both representations in
  memory are modified.  This introduces negligible overhead, as
  updating kernel tensors counts as a minuscule part of the total
  computation.  Having two copies, however, allows us to reuse the
  same algorithm for the forward and backward phases.

  The kernel update tensor ($\Delta W$) of Eq. (\ref{eq:update}) is
  stored only in the transposed form.

  Locating a memory array element in the linear address space is done
  by providing ``strides'' for each dimension and the location of the
  origin (address element at $[0][0]\dots[0]$).  A stride represents
  the linear distance in memory between two adjacent element along
  that dimension.  Given the strides, one can easily represent a
  sub--array by providing a new origin -- zeroth element of the
  sub--array.

  We will use a term sub--tensor to refer to a subset of a tensor with
  the same number or less dimensions.  The memory representation of a
  sub--tensor is then just represented as an additional pointer to the
  new origin inside the memory array representing the original array.
  Thus, any mention of sub--tensors and sub--arrays doesn't imply any
  memory copying.

\section{Forward and backward propagation}

  As highlighted in Section~\ref{sec-conv-layers}, the forward pass
  algorithm can be used for the backward pass if we (1) reflecting the
  kernels along all directions, thus converting convolution to
  cross--correlation, and (2) zero--pad the input image such that
  \texttt{valid} cross--correlation becomes \texttt{full}
  cross--correlation.

  We propose an algorithm for the forward pass that can handle an
  arbitrary zero padding of input image, thus the same algorithm can
  be used for the backward pass with no overhead.  As previously
  stated, we assume that two copies of kernels are stored - original
  and reflected.  This is reasonable because the memory required for
  kernels is typically much smaller than the memory required for the
  images.

  We will refer to the algorithm as the fwd-bwd algorithm.  Also,
  we'll refer to both the input/output images of the forward pass and
  the output/input gradients of the backward pass as
  \emph{input/output images}.

  \subsection{Algorithm overview}

  For our fwd--bwd algorithm, we will assume that both $F$ and $F'$
  are divisible by $S$, where $S$ is the width of the vector register.
  On AVX2 enabled CPUs, $S$ is 8, and on AVX512, $S$ equals to 16.
  This is reasonable as all state of the art networks for both
  registration and segmentation have both $F$ and $F'$ being a large
  multiple of $2$~\cite{krizhevsky2012imagenet, ronneberger2015u,
    simonyan2014very, sermanet2013overfeat, long2015fully,
    tran2015learning, ji20133d, maturana_iros_2015,
    maturana_icra_2014}.  An arbitrary number of images is supported
  by simply adding zero--initialized dummy images.  As both $F$ and
  $F'$ are relatively large, this would not introduce much
  computational overhead.  In further notation we will have $F =
  \alpha S$ and $F' = \alpha' S$.  In our public repository, however,
  we also provide the implementation for this specific case that has
  no overhead.

  Our algorithm consists of a stack primitives with increasing
  complexity.  Each more complex primitives uses the less complex ones
  to perform more complex computation.  Before describing the details
  of each primitive, we list the primitives in the increasing order of
  complexity together with main motivation and goal for each one.

  \begin{enumerate}
  \item {\bf Sub--image primitive} -- a primitive that computes $S^2$
    convolutions on $S$ small images producing $S$ small images of
    size $R_x, R_y, R_z$.  The goal of this primitive is to
    efficiently re--use data in the register file as well as data in
    L1 cache.  The techinque used by the primitive is known as
    register blocking.
  \item {\bf Full image primitive} -- a primitive that computes $S^2$
    convolutions on $S$ input images of an arbitrary size, producing
    $S$ output images.  The primitives optimally divides the
    computation into small blocks for which the previous primitive is
    used.  The division and order of computation is designed for
    efficient use of hardware pre--fetching.
  \item {\bf Sub--layer primitive} -- a primitive that computes
    $\beta S$ images by performing $\alpha \beta S^2$
    convolutions on $\alpha S$ images.  The primitive is designed to
    efficiently utilize higher levels of cache (L2, L3).
  \item {\bf Full layer primitive} -- parallelized and statically
    scheduled primitive that divides the computation into above
    primitive and assigns each available core one or more such
    primitives.  The goal is to evenly divide the computation among
    available cores.
  \end{enumerate}

  We proceed to describe each primitive in details.

  \begin{algorithm}
    {\footnotesize
      \begin{codebox}
        \Procname{$\proc{Fwd-Bwd-Subtask} \langle \Theta, R_x, R_y, R_x, K_x, K_y, K_z \rangle(i,w,o,\theta)$}
        \li \kw{simd float} $oreg[R_x][R_y][R_z]$
        \li \kw{simd float} $wreg$
        \li \If $\Theta$ \Comment Static if
        \li \Then $oreg[:][:][:][:] \gets \theta$ \Comment Load bias
        \li \Else
        \li       $oreg[:][:][:][:] \gets \proc{LOAD}(o[:][:][:][:])$
        \End \li \kw{end if}
        \li \kw{for} $k_x \gets 0 \To K_x-1$ \Comment Partially unrolled
        \li   \Do \kw{for} $k_y \gets 0 \To K_y-1$  \Comment Partially unrolled
        \li     \Do \kw{for} $f \gets 0 \To S - 1$ \Comment Partially unrolled
        \li       \Do \kw{for} $k_z \gets 0 \To K_z-1$  \Comment Conditionally unrolled
        \li         \Do $wreg \gets w[k_x][k_y][k_z][f][:]$
        \li \kw{for} $x \gets 0 \To R_x-1$ \Comment Fully unrolled
        \li   \Do \kw{for} $y \gets 0 \To R_y-1$ \Comment Fully unrolled
        \li      \Do \kw{for} $z \gets 0 \To R_z-1$ \Comment Fully unrolled
        \li         \Do $oreg[x][y][z][:] \gets \proc{FMADD}($
        \li       $wreg,$
        \li       $\proc{EXLOAD}(i[x+k_x][y+k_y][z+k_z][f]),$
        \li       $oreg[x][y][z][:])$
        \End \li \kw{end for} $z$
        \End \li \kw{end for} $y$
        \End \li \kw{end for} $x$
        \End \li \kw{end for} $k_z$
        \End \li \kw{end for} $f$
        \End \li \kw{end for} $k_y$
        \End \li \kw{end for} $k_x$
        \li $o[:][:][:][:] \gets \proc{STORE}(oreg[:][:][:][:])$
      \end{codebox}
    \caption{The finest granularity primitive that computes a
      sub--image of size $R_x \times R_y \times R_z$ of $S$ images by
      performing $S^2$ convolutions on $S$ input images with kernels
      of size $K_x \times K_y \times K_z$.}
    \label{alg:serial-forward-subtask}
    }
  \end{algorithm}

  {\bf Sub--image primitive} \quad The lowest level primitive is shown
  in Algorithm~\ref{alg:serial-forward-subtask}.  The arguments inside
  the angled brackets are known during compile--time, while the
  arguments inside the parenthesis are provided during run--time.
  This means that the same function with different compile--time
  arguments will be separately compiled and optimized by the compiler.
  With the loop bounds known during the compile time, the compiler is
  able to perform loop unrolling as commented in the pseudocode.  This
  is easily implemented using C++ templates.  Depending on the
  argument $\Theta$, the result of the convolution will be either
  accumulated to an image $o$, or the provided bias $\theta$ will be
  added to the result and the sum will be stored at $o$.

  Taking into account the data layout and Equation~\ref{eqn:forward},
  each value of the memory array $o[r_x][r_y][r_z][f']$ is computed
  via
  {\footnotesize
  \[
  \sum_{f} \sum_{k_x} \sum_{k_y} \sum_{k_z}
  i[r_x+k_x][r_y+k_y][r_z+k_z][f] \cdot w[k_x][k_y][k_z][f][f']
  \]
  } Our data layout allows for natural vectorization over $f'$.  Such
  that $S$ results $o[r_x][r_y][r_z][:]$ are computed by replacing the
  scalar product to a scalar--vector product
  {\footnotesize
  \[
  \sum_{f} \sum_{k_x} \sum_{k_y} \sum_{k_z}
  i[r_x+k_x][r_y+k_y][r_z+k_z][f] \cdot w[k_x][k_y][k_z][f][:]
  \]
  } We recognize that the values $w[k_x][k_y][k_z][f][:]$ are used for
  computing all $o[r_x][r_y][r_z][:]$ and reorder the computation such
  that once $w[k_x][k_y][k_z][:]$ is loaded (line 12) from memory, it
  is maximally re--used.

  Each iteration of the three innermost loops (lines 13-15) perform an
  independent multiple--add operation (lines 16--19).  The procedure
  $\proc{EXLOAD}$ loads a scalar value to all $S$ locations in a
  vector register.  The procedure $\proc{FMADD}$ is implemented as a
  CPU dependent macro, that either calls the fused multiple--add
  intrinsic, when available (AVX2, AVX512), or vector multiplication
  followed by a vector addition, when no fused multiple--add is
  supported by the CPU (AVX, SSE4).  AVX512 supports an instruction
  that performs fused multiple--add with the multiplication performed
  by an in--memory scalar, therefore the operation in lines(16--19) is
  transformed to a single instruction.

  To maximize the utilization, we want to increase the number of
  independent operations, thus prefer large values of $R_x \times R_y
  \times R_z$.  However, we need to make sure that all the computation
  is done using the register file -- $oreg$, $wreg$ and all possible
  auxiliary variables fit into the register file.  AVX512 CPUs have 32
  vector registers, and don't require auxiliary register for loading
  the scalar value, the maximal limit for $R_x \times R_y \times R_z$
  is 31.  SSE4, AVX and AVX2 capable CPUs require such auxiliary
  registers, thus the situation becomes a bit more complex.

  Note how the loop over $f$ comes before the loop over $k_z$.  As
  $k_z \ge 2$ in convolutional layers, we expect the compiler to
  unroll the loop over $k_z$ when auxiliary registers are required,
  and reorder the load and multiply--add operations such that the
  values loaded in the auxiliary registers are re--used.  As each
  auxiliary register is re--used approximately $k_z$ times, we need
  $k_z$ times less auxiliary registers than ones required by $oreg$.
  As $k_z \ge 2$ we choose to use $8$ registers for $oreg$.  The
  number of required auxiliary registers will be no greater than $4$.
  This allows for $8$ different independent operations.  This choice
  is very reasonable, as $8$ different independent operations allow
  for the full utilization of the CPU in the worst case -- having two
  FMA units with 4 cycle latency of operations.

  The working set of the primitive is expected to fit inside the $L1$
  cache, this is reasonable as the size of $R_x \times R_y \times R_z$
  is very small.  The cache miss rate can be thus approximated to
  $\frac{1}{C \times K_x \times K_y \times K_z}$, where $C$ is the
  number of floats that fit inside the cache--line.  This value is
  very low for even tiny kernels.  The regular memory access pattern
  (linear along $z$ dimension) allows for further reduction in cache
  misses due to hardware pre--fetching.  Measured cache hits averaged
  to over $98\%$ of kernel sizes of $1 \times 2 \times 2$ or larger,
  and any appropriate size of $R_x \times R_y \times R_z$.

  \begin{figure}
    \begin{center}
      \includegraphics[width=0.57\linewidth]{fig/division}
    \end{center}
    \caption{Computing sub--results of cross--correlation as
      cross--correlation of sub--images.}
    \label{fig:conv-division}
  \end{figure}

  {\bf Full image primitive} \quad Our next primitive performs $S^2$
  convolution on $S$ images of an arbitrary size, producing $S$ output
  images of size $I_z \times I_y \times I_x$.  This is done by
  dividing the output images into small sub--images
  (Fig~\ref{fig:conv-division} and using the primitive described
  above.  The size of each sub--image should be maximized, but is
  subject to constraints described above.  We prefer division into
  sub--images with large values of the least significant dimension,
  such that the adjacent pixels are adjacent in memory, thus benefit
  from hardware pre--fetching.  In most case, we will split the image
  into sub--images of size $1 \times 1 \times R$, where $R$ is the
  number of available registers as described above.  This is because
  the least significant dimension of the image is typically much
  larger than $R$.  Alternatively, we will try the subdivision into $1
  \times 2 \times \floor{R/2}$, etc...  Note that in most cases we
  will not be able to subdivide the image into equal parts.  This case
  is handled by instantiating extra primitives of smaller sizes, such
  that the image is fully covered.  This will result in the image
  covered in tiles, not all of them will have the same size.  However,
  we will have an optimally compiled code for each of them.

  The computation is then performed tile by tile, sliding along the
  least significant direction first, then the second least
  significant, etc...  This order of computation has multiple
  benefits.  First, it can benefit from hardware pre--fetching.
  Secondly, accessing contiguous data minimizes possible cache
  associativity conflicts, thus allowing us to store more data in the
  higher levels of cache, which can be reused as we slide along the
  more significant dimensions.

  {\bf Sub--layer primitive} \quad The next primitive considers all
  $\alpha S$ input images and computes the values of $\beta S$ ($\beta
  \le \alpha'$) output sub-images of size $I_z \times I_y \times I_x$
  by performing $\alpha \beta S^2$ convolutions.  This is achieved by
  invoking $\alpha \beta$ {\bf full image primitives} described above.

  \begin{figure}
    \begin{center}
      \includegraphics[width=0.8\linewidth]{fig/serialexec}
    \end{center}
    \caption{The recursive computation done by sub--layer primitive.
      Two levels of recursion are shown, each requiring less cache. }
    \label{fig:full-exec}
  \end{figure}

  This primitive is designed to efficiently use higher levels of cache
  when available, by ensuring that the order of {\bf full image
    primitives} are executed in a cache--friendly way.  This is
  achieved in a recursive fashion, where each successive recursive
  step requires half the memory of the previous step to fit the whole
  working set inside the cache.  Two levels of recursion are shown for
  the case of $\alpha = \beta = 4$ is shown on
  Fig.~\ref{fig:full-exec}.  In the first step (first row in
  Fig.~\ref{fig:full-exec}), both the number of input and output
  images are divided in half.  The next recursive step (second row in
  Fig.~\ref{fig:full-exec}) is then applied for all 4 possible
  subdivisions.  Note how the second level of recursion requires
  storage for 2 input and 2 output images in order to fit the whole
  working set inside cache, while the first level requires twice as
  much.  The order in which the 4 recursive problems are solved
  further increases cache reuse.  In the case when $\alpha \ge 2\beta$
  or $\beta \ge 2\alpha$ a simpler recursive step is performed, by
  dividing only the input/output images in half and solving two
  recursive sub--problems.

  {\bf Full layer primitive} \quad Our last primitive divides the
  computation of $B$ sets of $\alpha' S$ output images into sets of
  sub--layer primitives.  Each of $T$ available threads is statically
  scheduled to execute one or more such primitives.

  The main goal of the primitive is to equally divide the work among
  the $T$ threads such that once invoked they all finish roughly at
  the same time.  As described above, the $B$ sets of $\alpha'S$
  output images of size $D \times H \times W$ can regarded as a $5D$
  tensor of size $B \times (\alpha'S) \times D \times H \times W$.
  Again, we take a recursive approach to our static scheduling
  algorithm.  The input to our algorithm is a set of $T$ threads, and
  a $5D$ tensor of values that has to be computed.

  For the base case of our algorithm, when $T=1$, the values of the
  $5D$ tensor is scheduled on the given core to be executed by $B$
  sub--layer primitives described above.

  Our algorithm has two different recursive steps.  The first version
  of the recursive step first finds the smallest prime $p$ that
  divides $T$.  It then divides the set of $T$ threads into $p$ sets
  of $T/p$ threads, and the $5D$ tensor into $p$ equal sub--tensors by
  slicing it along the highest significant dimension that has a size
  of at least $p$.  It then recursively solves the $p$ problems for
  each set of $T/p$ threads and each equally sized sub--tensor.  If no
  dimension of the tensor is larger or equal than $p$, we apply the
  base case on an arbitrary available thread.  If the dimension of the
  tensor along which the splicing was performed is not divisible by
  $p$, another recursive problem is solved for all $T$ threads and the
  remaining sub--tensor.  Note that the size of the sub--tensor of
  each recursive sub--problem has at least $p$ times less elements.
  The rationale behind this step is that we will have each set of
  $T/p$ threads solving problems of exactly the same size.  As we
  expect them to be done roughly at the same time, when they can all
  work on computing the remaining values.

  As our previous primitives assume computation of multiple of $S$
  images, splicing along the second most significant dimension (of
  size $\alpha'S$) is done only when $\alpha' \ge p$.

  Note that in the recursive call for the remaining of the tensor, the
  value of $T$ doesn't decrease.  Thus, depth of recursion can reach
  up to $\log_2 E$, where $E$ is the number of elements in the $5D$
  tensor, which can be very large.  As the scheduling is done during
  compile--time, this can significantly increase the compile time, and
  create ``code bloat'', as primitives of many different sizes might
  have to be compiled.

  To prevent this we introduce the second version of the recursive
  step, which is applied when the number of elements of the $5D$
  tensor is smaller than $\epsilon$ times the number of elements of
  the original tensor.  In this case we divide the tensor along the
  most significant dimension larger than $T$ into $T$ sub--tensors,
  some of which are larger than others, after which $T$ base cases are
  applied and the work is scheduled on the $T$ threads.  The basic
  idea is that once computation required for the sub--tensor is small
  compared to the overall computation required for the layer, we can
  allow for some cores to do a bit more work than others without
  hurting overall performances.  In our implementation we picked
  $\epsilon$ to be $0.008$, which limits the recursion to maximal
  depth of $7$ while not allowing any single core to perform more then
  $1\%$ more work than other cores.

  \begin{figure}
    \begin{center}
      \includegraphics[width=0.67\linewidth]{fig/static2}
    \end{center}
    \caption{Computing the values of $F'=3$ channels of 2D images of
      size $4 \times 3$ using $T=2$ threads.  The three columns
      represent three stages.  The dark blue/red represent the values
      scheduled to be computed by $1^{st}/2^{nd}$ core.}
    \label{fig:problem-subdivision}
  \end{figure}

  An example of such scheduling for the special case of $B=1$, $F'=3$
  and $S=1$ and 2D images of size $4 \times 3$ is shown on
  Fig.~\ref{fig:problem-subdivision}.

  There are multiple benefits of our approach.  First, note that all
  the cores perform exactly the same computation.  When using multiple
  virtual threads per physical core, each thread can benefit from $L1$
  instruction cache.  Additionally, when multiple threads compute
  different images of the layer (as in the first step in
  Fig.~\ref{fig:problem-subdivision}), they access the elements of the
  input images in exactly the same order.  This will yield high hit
  rate of the higher levels of cache shared between cores.

  \subsection{Input image padding and strided convolution}

  As mentioned before, in order to be usable for the backward pass,
  our primitive has to support either implicit or explicit zero
  padding of input images.  When the input images are large, and
  kernels small, explicit zero padding of the input image only
  slightly increases the computational cost.  However, as the images
  get smaller this overhead can become significant.

  Additionally, once might consider a hybrid approach, where along
  some directions the input is explicitly padded, while along the
  other we perform implicit computation.

  We support implicit padding along an arbitrary of dimensions by
  modifying the lines 8--11 of
  Algorithm~\ref{alg:serial-forward-subtask}.  Instead of looping over
  all possible kernel offsets, for each invocation of the primitive,
  we provide additional runtime parameters, a pre--computed limits,
  for which the kernels offset are valid.

  Our algorithm can easily be modified to support strided convolution,
  which is supported in our implementation.  This is accomplished by
  simply modifying the line $18$ of
  Algorithm~\ref{alg:serial-forward-subtask} to account for the
  strides.

\section{Update phase}

  While the forward and backward pass perform cross--correlations and
  convolutions with a relatively large image with a small kernel
  producing another relatively large image, the cross--correlations
  performed during the update phase are performed on a large image
  with another large image resulting in a small image.

  The main difference between the cross--correlations performed during
  the fwd-bwd pass and the update phase is that

Similarly as in the fwd--bwd algorithm, our update phase algorithm
  consists of the following stack of primitives.

  Instead of computing
  \[
  \frac{\partial L}{\partial W_{j,i}} = \sum_{b=1}^B
  \frac{\partial L}{\partial I'_{b,j}}  \star I_{b,i}
  \]
  we compute
  \[
  \Bigg( \frac{\partial L}{\partial W_{j,i}} \Bigg)^T = \sum_{b=1}^B
  I_{b,i} \star \frac{\partial L}{\partial I'_{b,j}}
  \]

  That is the $GW^T$ tensor.

  In the update phase, each of the $F$ input images from the previous
  forward pass is cross--correlated with each of the $F'$ gradients
  with respect to the output obtained during the backward pass to
  produce $S^2$ gradients with respect to the kernel weights each of
  size $W_Z \times W_Y \times X$ -- the size of the kernel.  For
  simplicity, we will refer to the input images as simply ``images'',
  to the gradients with respect to the output as simply ``gradients'',
  and gradients with respect to the kernel weights as ``kernel
  gradients''.

  \begin{enumerate}
    \item {\bf Sub--kernel primitive} -- a primitive that computes a
      cross--correlation of each of $S$ images of size $R_Z \times R_Y
      \times (X + R_X - 1)$ with $S$ gradients of size $1 \times 1
      \times X$ to produce and accumulate the results of $S^2$ kernel
      gradients of size $R_Z \times R_X \times R_Y$.  As in the
      fwd--bwd case, this primitive is optimized for efficiently
      reusing the register file as well as $L1$ cache.
    \item {\bf Full kernel primitive} -- a primitive that computes
      cross--correlation of an arbitrary sized $S$ images with
      arbitrary sized $S$ gradients to produce $S^2$ kernel gradients.
    \item {\bf Sub--layer primitive} -- a primitive that computes
      $\alpha \times \beta \times S^2$ kernel gradients by
      cross--correlating each of $\alpha S$ images with $\beta S$
      gradients.
    \item {\bf Full layer primitive} -- parallelized primitive that
      divides the computation into a set of previous primitives and
      statically schedules execution.  As described below, this
      primitive might need to contain an extra parallelized reduction
      step, which is also statically scheduled.
  \end{enumerate}


  \begin{algorithm}
    {\footnotesize
      \begin{codebox}
        \Procname{$\proc{Update-Subtask} \langle R_x, R_y, R_z, Z, R_s \rangle(i,og,wg^T)$}
        \li \kw{simd register} $oreg[R_x][R_y][R_z][R_s]$
        \li \kw{simd register} $wreg$
        \li \For $s_0 \gets 0 \To S/R_s - 1$
        \li \Do $oreg[:][:][:][s_0R_s:s_0R_s+R_s-1][:] \gets$
        \li   \Do $\proc{LOAD}(wg^T[:][:][:][s_0R_s:s_0R_s+R_s-1][:])$
        \End
        \li \For $z_g \gets 0 \To Z-1$ \Comment Partially unrolled
        \li   \Do $wreg \gets \proc{LOAD}(og[1][1][z_g][:])$
        \li   \For $x \gets 0 \To R_x-1$ \Comment Fully unrolled
        \li   \Do \For $y \gets 0 \To R_y-1$  \Comment Fully unrolled
        \li   \Do \For $z \gets 0 \To R_z-1$  \Comment Fully unrolled
        \li   \Do \For $s_1 \gets 0 \To R_S-1$   \Comment Fully unrolled
        \li   \Do $oreg[x][y][z][s_0 \cdot R_s + s] \gets \proc{FMADD}($
        \li   \Do $wreg,$
        \li       $\proc{EXLOAD}(i[x][y][z+i][s_0 \cdot R_s + s]),$
        \li       $oreg[x][y][z][])$
        \End
        \End \li \kw{end for} $s$
        \End \li \kw{end for} $x$
        \End \li \kw{end for} $y$
        \End \li \kw{end for} $z$
        \End \li \kw{end for} $i$
        \li $wg^T[:][:][:][s_0R_s:s_0R_s+R_s-1][:] \gets$
        \li \Do $\proc{STORE}(oreg[:][:][:][s_0R_s:s_0R_s+R_s-1][:])$
        \End
        \End \li \kw{end for} $s_0$
      \end{codebox}
    \caption{Serial update subtask.}
    \label{alg:serial-update-subtask}
    }
  \end{algorithm}

  {\bf Sub--kernel primitive} \quad The lowest level primitive is
  shown in Algorithm~\ref{alg:serial-update-subtask}.  Following the
  same principles as in the fwd-bwd pass' sub--kernel primitive we
  vectorize the computation of $wg^T[r_x][r_y][r_z][f][f']$ computed
  via {\small
  \[
  \sum_{z}
  a[r_x][r_y][r_z+z][f] \cdot b[r_x][r_y][r_z][f']
  \]
  } such that the values of $wg^T[r_x][r_y][r_z][f][:]$ are computed
  via {\small
  \[
  \sum_{z}
  a[r_x][r_y][r_z+z][f] \cdot b[r_x][r_y][r_z][:]
  \]
  } Again, we recognize the reuse of $og[r_x][r_y][r_z][:]$ and
  re--order the loops appropriately.  In order to allow in--register
  computation, same constraints are imposed on $R_x \times R_y \times
  R_z \times R_s$ -- Maximal of $31$ for AVX512 and $8$ for SSE4, AVX
  and AVX2.  Additionally we allow only values for $R_s$ that divide
  $S$.  In order for the working set to fit inside the $L1$ cache, $Z$
  should be sufficiently small.  The number of bytes required for the
  working set equals $4S(R_xR_y+1)(R_z+Z-1)$.  For a given choice of
  $R_x, R_y$ and $R_z$, and given size of $L1$ cache, we
  conservatively choose $Z$ so that no more than half the cache is
  required for the working set.

   \begin{figure}
     \centering
     \includegraphics[width=0.99\linewidth]{fig/update2}
     \caption{An example of the {\bf full kernel primitive} and {\bf
         full gradient primitive} for the special case of $S=1$,...}
     \label{fig:conv-decomposition}
   \end{figure}

  {\bf Full kernel primitive} computes $S^2$ kernel gradients by
  cross--correlating each of $S$ images with $S$ gradients.  The
  computation is performed in two steps.  First, the gradients are
  split into sub--gradients of size $1 \times 1 \times Z$.  The result
  is then obtained as the sum of cross--correlations of each of the
  sub--gradients with an appropriate sub--image, as depicted on
  Fig~\ref{conv-decomposition} (middle column).  The
  cross--correlation of each sub--gradient is further split into
  sub--kernel primitives (right column on
  Fig.~\ref{conv-decomposition}.  We choose the $R_x, R_y, R_z$ and
  $R_s$ to maximize the register--file utilization, but subject to the
  limits described above.  Further, we require that $R_x, R_y$ and
  $R_z$ divide $K_x, K_y$ and $K_z$ respectively.  Note that this can
  always be accomplished be setting $R_x=R_y=R_z=1$ and $R_s=S$.
  However this choice might not be optimal.  For example, when $K_z=3$
  on AVX512 CPU, we could pick $R_s=8$ and $R_z=3$, which utilizes
  $24$ registers, which is better than choosing $R_z=1$ and
  $R_s=S=16$.  The optimal division is the one for which $R_x \times
  R_y \times R_z \times R_s$ is maximized.  When more combinations are
  possible, we prefer ones with large values of $R_s$, then $R_z$
  etc...  The value for $Z$ is then obtained using as described above.

  The computation is performed by iterating over the least significant
  dimension, then second least significant, etc..

  {\bf Sub--layer primitive} performs $\beta S \beta' S$
  cross--correlation.  Each of $\beta S$ input images is
  cross--correlated with each of $\beta'S$ output gradients to produce
  $\beta S \beta' S$ kernel gradients.  This primitive is designed for
  maximal reuse of higher levels of cache.  To achieve that, the
  computation is performed in the same fashion as in the fwd--bwd
  sub--layer primitive (Fig.~\ref{fig:full-exec}).

  {\bf Full layer primitive} \quad The main goal of the primitive is
  to parallelize the computation over $T$ available threads.  In
  contrast to the fwd--bwd full layer primitive where each thread can
  execute multiple sub--layer primitives, here we assign exactly one
  sub--layer primitive to each thread.  Each thread then executes the
  primitive on the subset of $B$ input--image output--gradient pairs.

  The scheduling is done in two steps.  Let $P =
  \proc{GCD}(T,\alpha\alpha')$, we first select $\beta$ and $\beta'$
  such that they divide $\alpha$ and $\alpha'$ respectively, and
  $\beta\beta'P = T$.  Out of all possible values we pick the one for
  which $|\beta -\beta'|$ is minimized.  The threads are then also
  divided into $P$ sets of $T'=T/P$ threads, each of which will
  compute $\beta\beta'S^2$ kernel gradients.

  The values of the kernel gradient tensor $GW^T(A
  \beta':A\beta'+\beta'-1,B \beta: B\beta + \beta-1, :,:,:)$ is
  performed by the $A + B \alpha / \beta$-th set of $T'$ threads.

  In the second step, we consider each of the $P$ sets of $T'$
  threads.  Kernel gradients to be computed by each subset is computed
  by cross--correlating each of $\beta S$ input images with $\beta'S$
  output gradients for each set of inputs in the batch of size $B$.

  Further sub--division can be accomplished by dividing the
  computation such that different threads process either different
  sets in the batch or to process sub--images as depicted in the
  middle column of Fig.~\ref{conv-decomposition}.  The final result is
  then obtained by accumulating the result obtained by each thread.
  This means that an additional ``reduction'' pass is necessary -- we
  need to accumulate $T'$ results obtained by each of the threads.  As
  we need to divide the problem into exactly $T'$ subproblems, the
  recursive scheduling approach used for the fwd-bwd pass is not
  possible.  Instead, to divide the problem as evenly as possible over
  the $T'$ threads we perform the following recursive approach, that
  is not guaranteed to divide the problem as evenly as in the fwd--bwd
  case.

\section{Experiments}

  In order to support the claim that our algorithm is agnostic to the
  ConvNet architecture as well as hardware it is running on decide to
  benchmark the convolutional layers of state of the art, top
  performing networks used for -- (1) 2D object detection, (2) 2D
  image segmentation and (3) spatiotemporal feature learning (3D)

  \begin{table} \centering
    \setlength\tabcolsep{2.5pt}
    \begin{tabular}{cr !{\vrule width0.8pt} cccccc  }
      &  & B & F & F' & Image Size & Padding & Kernel Size  \\
      \hline
      \multirow{5}{*}{\rotatebox{90}{\textbf{VGG-a}}}
      & C2 & 64  & 64  &  128 & $\angled{112,112}$ & $\angled{1,1}$ & $\angled{3,3}$ \\
      & C3 & 64  & 128 &  256 & $\angled{56,56}$   & $\angled{1,1}$ & $\angled{3,3}$ \\
      & C4 & 64  & 256 &  256 & $\angled{56,56}$   & $\angled{1,1}$ & $\angled{3,3}$ \\
      & C5 & 64  & 256 &  512 & $\angled{28,28}$   & $\angled{1,1}$ & $\angled{3,3}$ \\
      & C6 & 64  & 512 &  512 & $\angled{28,28}$   & $\angled{1,1}$ & $\angled{3,3}$ \\
      \hline
      \multirow{5}{*}{\rotatebox{90}{\textbf{U-Net}}}
      & C1b & 1  & 64  &  64 & $\angled{570,570}$  & $\angled{0,0}$ & $\angled{3,3}$ \\
      & C2b & 1  & 128 &  128 & $\angled{282,282}$ & $\angled{0,0}$ & $\angled{3,3}$ \\
      & C3b & 1  & 256 &  256 & $\angled{138,138}$ & $\angled{0,0}$ & $\angled{3,3}$ \\
      & C4b & 1  & 512 &  512 & $\angled{66,66}$   & $\angled{0,0}$ & $\angled{3,3}$ \\
      & C5b & 1  & 1024 &  1024 & $\angled{30,30}$ & $\angled{0,0}$ & $\angled{3,3}$ \\
      \hline
      \multirow{5}{*}{\rotatebox{90}{\textbf{VGG-a}}}
      & C2a & 32  & 64  &  128 & $\angled{16,56,56}$ & $\angled{1,1,1}$ & $\angled{3,3,3}$ \\
      & C3a & 32  & 128 &  256 & $\angled{8,28,28}$ & $\angled{1,1,1}$ & $\angled{3,3,3}$ \\
      & C3b & 32  & 256 &  256 & $\angled{8,28,28}$ & $\angled{1,1,1}$ & $\angled{3,3,3}$ \\
      & C4a & 32  & 256 &  512 & $\angled{4,14,14}$ & $\angled{1,1,1}$ & $\angled{3,3,3}$ \\
      & C4b & 32  & 512 &  512 & $\angled{4,14,14}$ & $\angled{1,1,1}$ & $\angled{3,3,3}$ \\
      \hline
      \multirow{5}{*}{\rotatebox{90}{\textbf{Toy}}}
      & 2D1 & 64  & 48 &  96 & $\angled{114,114}$ & $\angled{0,0}$ & $\angled{3,3}$ \\
      & 2D2 & 64  & 48 &  96 & $\angled{58,58}$ & $\angled{0,0}$ & $\angled{5,5}$ \\
      & 2D3 & 64  & 48 &  96 & $\angled{58,58}$ & $\angled{0,0}$ & $\angled{11,11}$ \\
      & 3D1 & 32  & 48 &  96 & $\angled{10,30,30}$ & $\angled{0,0}$ & $\angled{2,3,3}$ \\
      & 3D2 & 32  & 48 &  96 & $\angled{10,20,20}$ & $\angled{0,0}$ & $\angled{3,5,5}$ \\
      \hline
    \end{tabular}
    \caption{Benchmarked ConvNet layers.}
  \end{table}

  For the 2D object detection, we decide on VGG-A version of
  OxfordNet~\cite{simonyan2014very} as one of the
  ImageNet~\cite{imagenet_cvpr09,ILSVRC15}, winners that is also
  commonly used for benchmarking different ConvNet
  algorithms~\cite{imagenetwinners}.  For 2D image segmentation we
  choose U--Net~\cite{ronneberger2015u}, a top performing network for
  biomedical image segmentation.  The main difference between object
  detection and segmentation Convnets is that the object detection
  networks use ``batch training'' (multiple sets of inputs at the
  time) with gradually downsampled images.  On the other side,
  segmentation networks use fairly large images with $B=1$.  As for
  our 3D network, we choose C3D~\cite{maturana_iros_2015} a state of
  the art network for spatiotemporal feature learning.  For all the
  networks, we focus on 5 most computationally expensive convolutional
  layers, shown on Table~\ref{table:layers}.  As all chosen ConvNets
  have kernel sizes of $3 \times 3$ or $3^3$ in the 3D case, we
  benchmark some different kernel sizes.

  We include layers with non--traditional number of images (not power
  of two).  For the 2D case we include two extra kernel sizes of $5
  \times 5$ and $11 \times 11$, inspired by
  AlexNet~\cite{krizhevsky2012imagenet}.  For the 3D case, we include
  non--isotropic kernel size of $2 \times 3 \times 3$, inspired by
  VD3D3D~\cite{lee2015recursive}, as well as one of $3 \times 5 \times
  5$.


  \begin{table}
    \begin{center}
      \setlength\tabcolsep{2.5pt}
      \begin{tabular}{lrrrr}
        \toprule
        CPU & Frequency & CPUs $\times$ Cores/Threads & GFLOPS\\
        \midrule
        i7-6700K (Skylake) & 4GHZ & 1 $\times$ 4/8 & 512\\
        $4\times$ E7-8890v3 (Haswell) & 2.5GHz & 4 $\times$ 18/36 & 5760\\
        Xeon Phi 7210 & 1.1GHz & 1 $\times$ 64/256 & 4505.6\\
        \toprule
        GPU & Frequency & CUDA cores & GFLOPS\\
        \midrule
        Titan X (Maxwell) & 1GHz & 3072 & 6600\\
        Titan X (Pascal) & 1.5GHz & 3584  &  11000\\
        \bottomrule
      \end{tabular}
    \end{center}
    \caption{Machines used for the benchmarks}
  \end{table}

  In addition, we benchmark each of the networks on three generatiopns
  of Intel Xeon processors (Table~\ref{table:cpus}).  All the machines
  except the Skylake CPU were set to run at constant frequency (we did
  not have root access to the Skylake machine).  In addition, we have
  benchmarked the 3D network on the two latest generations of the
  Titan X GPU.

  \subsection{CPU utilization and scalability}


  \begin{figure*} \centering
    \small
    \setlength\tabcolsep{0.5pt}
    \begin{tabular}{ >{\centering\arraybackslash}c ccccccl }
      \toprule
      & \multicolumn{2}{c}{\textbf{Skylake}}
      & \multicolumn{2}{c}{\textbf{Haswell}}
      & \multicolumn{2}{c}{\textbf{Knights Landing}} & \\
      \midrule
      & Forward & Update & Forward & Update & Forward & Update & \\
      \midrule
      \rotatebox{90}{\qquad \textbf{VGG-A}}
      & \includegraphics[height=2.4cm]{fig/vgg-fwd-skylake}
      & \includegraphics[trim=8mm 0mm 0mm 0mm,clip,height=2.4cm]{fig/vgg-upd-skylake}
      & \includegraphics[trim=8mm 0mm 0mm 0mm,clip,height=2.4cm]{fig/vgg-fwd-haswell}
      & \includegraphics[trim=8mm 0mm 0mm 0mm,clip,height=2.4cm]{fig/vgg-upd-haswell}
      & \includegraphics[trim=8mm 0mm 0mm 0mm,clip,height=2.4cm]{fig/vgg-fwd-knl}
      & \includegraphics[trim=8mm 0mm 0mm 0mm,clip,height=2.4cm]{fig/vgg-upd-knl}
      & \includegraphics[height=2.4cm]{fig/vgg-legend} \\
      \midrule
      \rotatebox{90}{\qquad \textbf{U-Net}}
      & \includegraphics[height=2.4cm]{fig/unet-fwd-skylake}
      & \includegraphics[trim=8mm 0mm 0mm 0mm,clip,height=2.4cm]{fig/unet-upd-skylake}
      & \includegraphics[trim=8mm 0mm 0mm 0mm,clip,height=2.4cm]{fig/unet-fwd-haswell}
      & \includegraphics[trim=8mm 0mm 0mm 0mm,clip,height=2.4cm]{fig/unet-upd-haswell}
      & \includegraphics[trim=8mm 0mm 0mm 0mm,clip,height=2.4cm]{fig/unet-fwd-knl}
      & \includegraphics[trim=8mm 0mm 0mm 0mm,clip,height=2.4cm]{fig/unet-upd-knl}
      & \includegraphics[height=2.4cm]{fig/unet-legend} \\
      \midrule
      \rotatebox{90}{\qquad \textbf{C3D}}
      & \includegraphics[height=2.4cm]{fig/d3d-fwd-skylake}
      & \includegraphics[trim=8mm 0mm 0mm 0mm,clip,height=2.4cm]{fig/d3d-upd-skylake}
      & \includegraphics[trim=8mm 0mm 0mm 0mm,clip,height=2.4cm]{fig/d3d-fwd-haswell}
      & \includegraphics[trim=8mm 0mm 0mm 0mm,clip,height=2.4cm]{fig/d3d-upd-haswell}
      & \includegraphics[trim=8mm 0mm 0mm 0mm,clip,height=2.4cm]{fig/d3d-fwd-knl}
      & \includegraphics[trim=8mm 0mm 0mm 0mm,clip,height=2.4cm]{fig/d3d-upd-knl}
      & \includegraphics[height=2.4cm]{fig/d3d-legend} \\
      \midrule
      \rotatebox{90}{\qquad \textbf{Toy}}
      & \includegraphics[height=2.4cm]{fig/toy-fwd-skylake}
      & \includegraphics[trim=8mm 0mm 0mm 0mm,clip,height=2.4cm]{fig/toy-upd-skylake}
      & \includegraphics[trim=8mm 0mm 0mm 0mm,clip,height=2.4cm]{fig/toy-fwd-haswell}
      & \includegraphics[trim=8mm 0mm 0mm 0mm,clip,height=2.4cm]{fig/toy-upd-haswell}
      & \includegraphics[trim=8mm 0mm 0mm 0mm,clip,height=2.4cm]{fig/toy-fwd-knl}
      & \includegraphics[trim=8mm 0mm 0mm 0mm,clip,height=2.4cm]{fig/toy-upd-knl}
      & \includegraphics[height=2.4cm]{fig/toy-legend} \\
      \bottomrule

    \end{tabular}
    \caption{Scalability}
  \end{figure*}

  First, we demonstrate the near--linear scalability of our approach,
  regardless of the layer shape.  We show that our approach has the
  same utilization for all of them.

  Secondly, we compare the speed of our approach to alternative 2D
  primitives -- specifically CcT, MKL--DNN and MKL--2017.  We show
  that our approach is competitive, over--performing the next bet
  competitor by a small margin on Xeon and under--performing by a
  small margin on the KNL for 2D object--detection.  However, for
  semantical segmentation, we greatly outperform all competitors on
  both Xeon and Xeon Phi.

  Finally, we compare the speed of 3D networks to state of the art 3D
  primitives for the GPU.


  \begin{table} \centering
    \setlength\tabcolsep{2.5pt}
    \begin{tabular}{cr !{\vrule width0.8pt} rr|rr !{\vrule width0.8pt} rr|rr  }
      & & \multicolumn{4}{c !{\vrule width0.8pt} }{\textbf{cuDNNv5 (Titan X)}} & \multicolumn{4}{c}{\textbf{ZNNphi (Xeon)}} \\
      & & \multicolumn{2}{c|}{Maxwell} & \multicolumn{2}{c!{\vrule width0.8pt}}{Pascal}
      & \multicolumn{2}{c|}{Haswell} & \multicolumn{2}{c}{KNL} \\
      &  & Fwd & Bwd & Fwd & Bwd& Fwd & Bwd& Fwd & Bwd \\
      \hline
      \multirow{5}{*}{\rotatebox{90}{\textbf{C3D}}}
      & C2a & 186.8 & 370.3 & 168.0 & 336.0 & {\bf 147.6} & {\bf 325.2} & 218.1 & 456.3  \\
      & C3a & 90.1  & 180.0 & 78.5  & {\bf 161.2} & {\bf 72.2}  & 162.9 & 112.3 & 234.3  \\
      & C3b & 178.0 & 359.5 & 153.0 & 310.0 & {\bf 141.0} & {\bf 308.1} & 222.4 & 467.1  \\
      & C4a & 32.8  & 68.7  & 28.3  & {\bf 59.6}  & {\bf 27.7}  & 63.1  & 43.4  &  98.8  \\
      & C4b & 65.1  & 135.1 & 55.7  & {\bf 117.4} & {\bf 55.2}  & 131.2 & 85.3  & 196.1  \\
      \hline
    \end{tabular}
    \caption{3D vs GPU}
  \end{table}

  \begin{table} \centering
    \setlength\tabcolsep{2.5pt}
    \begin{tabular}{cr !{\vrule width0.8pt} rr|rr|rr|rr  }
      & & \multicolumn{2}{c|}{MKL-DNN} & \multicolumn{2}{c|}{MKL-2017}
      & \multicolumn{2}{c|}{CcT} & \multicolumn{2}{c}{ZNNPhi} \\
      &  & Fwd & Bwd & Fwd & Bwd& Fwd & Bwd& Fwd & Bwd \\
      \hline
      \multirow{5}{*}{\rotatebox{90}{\textbf{VGG-A}}}
      & C2  & 79.1  & 196.4 & 43.8 & 99.3  & 141.2 & 317.8 & {\bf 33.4} & {\bf 60.4}  \\
      & C3  & 42.4  & 106.8 & 30.0 & 78.5  & 118.5 & 194.3 & {\bf 24.5} & {\bf 60.3}  \\
      & C4  & 101.3 & 322.3 & 58.6 & 158.0 & 263.5 & 395.4 & {\bf 48.2} & {\bf 100.6} \\
      & C5  & 44.2  & 203.5 & 27.3 & 76.2  & 92.6  & 233.9 & {\bf 24.2} & {\bf 53.2}  \\
      & C6  & 89.8  & 466.5 & 54.8 & 149.9 & 196.8 & 466.3 & {\bf 47.2} & {\bf 104.2} \\
      \hline
      \multirow{5}{*}{\rotatebox{90}{\textbf{U-Net}}}
      & C1b  & 257.44 & 494.93 & 183.16 & 195.86 & 1739 &  542 & {\bf 9.02} & {\bf 18.12} \\
%%      & C2a  & 66.07  & 129.50 & 43.92  & 38.38  & 641  & 1199 & {\bf 3.80} & {\bf  8.51} \\
      & C2b  & 127.55 & 248.83 & 82.80  & 42.53  & 1052 & 1671 & {\bf 6.07} & {\bf 13.91} \\
%%      & C3a  & 36.47  & 67.78  & 21.64  & 13.08  & 1032 & 1627 & {\bf 3.55} & {\bf  8.88} \\
      & C3b  & 66.17  & 126.35 & 38.48  & 25.20  & 2988 & 4645 & {\bf 5.47} & {\bf 13.41} \\
%%      & C4a  & 21.04  & 40.69  & 11.50  & 13.80  & 693  & 1244 & {\bf 3.46} & {\bf  8.62} \\
      & C4b  & 53.89  & 90.54  & 18.66  & 26.90  & 2172 & 4033 & {\bf 5.16} & {\bf 13.21} \\
%%      & C5a  & 11.49  & 36.59  & 5.38   & 9.98   & 899  & 1754 & {\bf 3.23} & {\bf  8.12} \\
      & C5b  & 19.34  & 77.86  & 9.24   & 22.09  & 1667 & 3198 & {\bf 4.35} & {\bf 11.23} \\
      \hline

    \end{tabular}
    \caption{Haswell}
  \end{table}


  \begin{table} \centering
    \setlength\tabcolsep{2.5pt}
    \begin{tabular}{cr !{\vrule width0.8pt} rr|rr|rr  }
      & & \multicolumn{2}{c|}{MKL-DNN} & \multicolumn{2}{c|}{MKL-2017}
      & \multicolumn{2}{c}{ZNNPhi} \\
      &  & Fwd & Bwd& Fwd & Bwd& Fwd & Bwd \\
      \hline
      \multirow{5}{*}{\rotatebox{90}{\textbf{VGG-A}}}
      & C2  & 101.9 & 266.8 & {\bf 34.6} & 101.8 & 40.1 & {\bf  80.7}  \\
      & C3  & 72.5  & 192.3 & {\bf 32.9} & 87.1  & 39.7 & {\bf  76.8}  \\
      & C4  & 142.0 & 402.9 & {\bf 65.4} & 166.2 & 80.6 & {\bf 158.9}  \\
      & C5  & 53.0  & 268.5 & {\bf 32.8} & {\bf 72.9}  & 40.1 & 77.5   \\
      & C6  & 108.3 & 557.3 & {\bf 64.8} & {\bf 172.3} & 78.6 & 157.3  \\
      \hline
      \multirow{5}{*}{\rotatebox{90}{\textbf{U-Net}}}
      & C1b  & 799.60 & 1603.93 & 141.01 & 61.96 & {\bf 9.89} & {\bf 21.81} \\
%%      & C2a  & 195.82 & 410.73  & 41.90  & 19.20 & {\bf 5.27} & {\bf 11.19} \\
      & C2b  & 382.60 & 790.76  & 48.93  & 23.27 & {\bf 9.79} & {\bf 20.31} \\
%%      & C3a  & 96.19  & 212.08  & 11.97  & 10.01 & {\bf 5.13} & {\bf  9.77} \\
      & C3b  & 187.08 & 421.61  & 23.10  & 16.84 & {\bf 8.50} & {\bf 16.71} \\
%%      & C4a  & 48.05  & 110.39  & 5.60   & 23.26 & {\bf 4.44} & {\bf  9.32} \\
      & C4b  & 94.81  & 259.73  & 10.56  & 24.70 & {\bf 7.31} & {\bf 15.49} \\
%%      & C5a  & 25.17  & 90.04   & {\bf 3.66}   & 8.70  & 3.81 & {\bf  8.34} \\
      & C5b  & 44.01  & 167.88  & {\bf 5.45}   & {\bf 10.84} & 6.01 & 12.33 \\
      \hline

    \end{tabular}
    \caption{KNL}
  \end{table}

\section{Conclusion and closing remarks}

  Based on the experiments conducted, we conclude that our
  compile--time meta--programming approach to optimize ConvNets yields
  very high utilization and high scalability over large number of
  cores regardless of the ConvNet type.

  This approach is very easy to implement using C++ templates.  Our
  implementation, which is publicly available at
  \url{https://github.com/seung-lab/znnphi-release}, consist of around
  1000 lines of C++ code for each of two algorithms.  The
  implementation, however, is not meant to be used as a final product.
  Future work will include creating a production ready code, as well
  as integrating the primitives into popular ConvNet frameworks such
  as Caffe.

  Additional benchmarks on AVX and SSE4 CPUs, achieving roughly the
  same utilization, were performed but not reported as these older
  generations of CPUs were capable of much less FLOPS.  In practice,
  such CPUs would never be used for training ConvNets.  However,
  consumer applications that use already trained ConvNets, such as
  image processing apps, can greatly benefit from our approach by
  efficiently running the forward propagation on older CPU
  generations.

  Finally, sliding window inference, which is important for
  segmentation problems, can greatly benefit from our approach as
  access to more RAM can greatly increase the inference
  throughput~\cite{zlateski2016znni}.  In this cases we expect our
  approach to over--perfom the GPUs, even for the 2D cases, as
  generally CPUs have access to much more RAM.


%\clearpage
\bibliographystyle{./IEEEtran/IEEEtranBST/IEEEtran}
\bibliography{IEEEabrv,znnphi}
\end{document}
